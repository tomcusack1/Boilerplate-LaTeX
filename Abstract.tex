\chapter*{Abstract}

\begin{tabular}{@{}ll@{}}
{\bfseries Title}   & Practical improvements to the deformation method for point counting \\
{\bfseries Name}    & Sebastian Pancratz \\
{\bfseries College} & Hertford College \\
{\bfseries Degree}  & Doctor of Philosophy in Mathematics \\
{\bfseries Term}    & Trinity~2012
\end{tabular}

\bigskip

\noindent
In this thesis we investigate practical aspects related 
to point counting problems on algebraic varieties over finite fields. 
In particular, we present significant improvements to Lauder's 
deformation method for smooth projective hypersurfaces, which 
allow this method to be successfully applied to previously 
intractable instances.

{Part~I} is dedicated to the deformation method, including a complete 
description of the algorithm but focussing on aspects for which we 
contribute original improvements.  In Chapter~\ref{ch:01-DiagFrob} 
we describe the computation of the action of Frobenius on the rigid 
cohomology space associated to a diagonal hypersurface; 
in Chapter~\ref{ch:GMConnection} we develop a method for fast 
computations in the de~Rham cohomology spaces associated to the family, 
which allows us to compute the Gauss--Manin connection matrix.  
We conclude this part with a small selection of examples in 
Chapter~\ref{ch:01-Exmp}.

In {Part~II} we present an improvement to Lauder's fibration method. 
We manage to resolve the bottleneck in previous computation, which is 
formed by so-called polynomial radix conversions, employing power 
series inverses and a more efficient implementation.

Finally, {Part~III} is dedicated to a comprehensive treatment of 
the arithmetic in unramified extensions of~$\mathbf{Q}_p$, which is 
connected to the previous parts where our computations rely on efficient 
implementations of $p$-adic arithmetic.  We have made these routines 
available for others in {\sc{FLINT}} as individual modules for $p$-adic 
arithmetic.


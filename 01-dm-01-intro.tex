% Chapter:  Introduction to point counting  %%%%%%%%%%%%%%%%%%%%%%%%%%%%%%%%%%%

% Varieties over finite fields and zeta functions %%%%%%%%%%%%%%%%%%%%%%%%%%%%%

\section{Varieties over finite fields and zeta functions}

Let $X$ denote an algebraic variety over the finite field~$\mathbf{F}_q$ 
with $q = p^a$ elements.  In particular, $X$ is also defined 
over every algebraic extension $\mathbf{F}_{q^i}$, for $i \geq 0$, 
and we can let $N_i = \abs{ X(\mathbf{F}_{q^i}) }$ denote the number 
of points of $X$ over $\mathbf{F}_{q^i}$.  The \emph{zeta function} 
of $X$ is then defined as the formal power series in $\mathbf{Q}[[T]]$,
\begin{equation*}
Z(X, T) = \exp \Bigl( \sum_{i=0}^{\infty} \frac{N_i T^i}{i} \Bigr).
\end{equation*}
In the above form, the power series~$Z(X, T)$ is represented by an 
infinite series and it is unclear whether this can be explicitly 
computed by an algorithm viz.\ a \emph{finite} routine.  This issue 
was resolved by Dwork~\citep{Dwork1960} in 1960:

\begin{thm}
Let $X$ be an algebraic variety over the finite field~$\mathbf{F}_q$.  
Then the zeta function $Z(X, T)$ is a quotient of two polynomials with 
integer coefficients.
\end{thm}

Moreover, Bombieri~\citep{Bombieri1978} provided an explicit bound 
for the total degree of the rational function representing $Z(X, T)$.
This implies that we can, at least in principle, compute the zeta 
function by computing a finite number of the quantities $N_i$ by 
directly enumerating the points $\abs{X\bigl(\mathbf{F}_{q^i}\bigr)}$.
Dwork's theorem is included in a group of results known as the 
\emph{Weil conjectures}, statements conjectured by Weil~\citep{Weil1949} 
in~1949 which have subsequently been proved by Dwork, Grothendieck, 
Deligne, et al.\ (see~\citep[Appendix~C]{Har77}).  We follow 
Kedlaya~\citep[Theorem~1.2.1]{Kedlaya2011} in the statement of the Weil 
conjectures:

\begin{thm} \label{thm:01-Zetafunctions}
Let $X$ be a separated scheme of finite type over the finite 
field~$\mathbf{F}_q$.  Then the zeta function of~$X$ is the power series 
representation of a rational function~$T$.  Moreover, if $X$ is smooth 
and proper over~$\mathbf{F}_q$, then there is a unique way to write 
\begin{equation}
Z(X, T) = \prod_{i=0}^{2 \dim(X)} P_i(T)^{(-1)^{i+1}}
\end{equation}
for some polynomials $P_i(T) \in \mathbf{Z}[T]$ with $P_i(0) = 1$, 
satisfying the following conditions.
\begin{enumerate}
\item We have 
\begin{equation*}
P_i(q^{-i} / T) = \pm q^{-i \deg(P_i) / 2} T^{- \deg(P_i)} P_i(T),
\end{equation*}
with the sign being $+$ whenever $i$ is odd.  In other words, the roots 
of~$P_i$ are invariant under the map $r \mapsto q^{-i} / r$, and if $i$ 
is odd then the multiplicities of $\pm q^{-i/2}$ are even.
\item The complex roots of $P_i$ all have absolute value $q^{-i/2}$.
\item If $X \cong \mathfrak{X}_{\mathbf{F}_q}$ for some smooth proper 
scheme~$\mathfrak{X}$ over the local ring $R = \mathfrak{o}_{K,\mathfrak{p}}$, 
for some number field~$K$ and some prime ideal~$\mathfrak{p}$ of 
$\mathfrak{o}_K$ with residue field~$\mathbf{F}_q$, then for any 
embedding~$K \into \mathbf{C}$, 
\begin{equation*}
\deg(P_i) = \dim_{\mathbf{C}} H^{i} \bigl( (\mathfrak{X} \times_R \mathbf{C})^{\text{an}}, \mathbf{C} \bigr).
\end{equation*}
In other words, $\deg(P_i)$ equals the $i$-th Betti number of 
$\mathfrak{X} \times_{R} \mathbf{C}$.
\end{enumerate}
\end{thm}

A discussion of the proof of this theorem is significantly 
beyond the scope of this thesis.  As a starting point for 
further details, we refer the reader to 
Hartshorne~\citep[Appendix~C]{Har77} and Osserman~\citep{Osserman2008}.

Dwork's theorem on the rationality of the zeta function implies 
that $Z(X, T)$ is already defined by a finite initial segment of 
the sequence~$(N_i)_{i \geq 0}$.  In particular, the problem of 
computing the zeta function can be formulated as follows:

\begin{prob} \label{prob:dm-pointcounting}
Given an algebraic variety~$X$ defined by a finite list of 
multivariate polynomials in~$\mathbf{F}_{q}[x_0, \dotsc, x_n]$, 
efficiently compute the rational function in~$\mathbf{Q}(T)$ 
representing $Z(X, T)$.
\end{prob}

% Point counting algorithms %%%%%%%%%%%%%%%%%%%%%%%%%%%%%%%%%%%%%%%%%%%%%%%%%%%

\section{Point counting algorithms}

Computational routines addressing Problem~\ref{prob:dm-pointcounting} are 
known as \emph{point counting} algorithms.  Perhaps the most naive approach 
is to explicitly exhibit the points of~$X$ over~$\mathbf{F}_{q^i}$ by an 
exhaustive search for sufficiently many $i \geq 0$, determining $Z(X, T)$.  
For example, in the case of an elliptic curve~$E$ over a finite 
field~$\mathbf{F}_{q}$, the zeta function is given by 
\begin{equation*}
Z(E, T) = \frac{1 - a_q T + q T^2}{(1 - T)(1 - q T)}
\end{equation*}
where $a_q = q + 1 - \abs{E(\mathbf{F}_{q})}$.  However, this 
approach is only feasible for the smallest of instances of 
Problem~\ref{prob:dm-pointcounting}.  More sophisticated algorithms 
have been developed, and we refer the reader to surveys by 
Chambert-Loir~\citep{ChambertLoir2008} on point counting algorithms 
in general and Kedlaya~\citep{Kedlaya2004} on $p$-adic methods in 
particular.  In the following, we only mention a few different approaches.

In~1985, Schoof~\citep{Schoof1985} proposed an $\ell$-adic algorithm for 
computing the zeta function of an elliptic curve~$E$.  The main idea behind 
this approach is to compute the trace~$a_q \bmod{\ell}$ of the Frobenius 
endomorphism on $\Het^{1}(E, \mathbf{F}_{\ell})$ for sufficiently 
many primes~$\ell$ and then determine the integer~$a_q$ using the Chinese 
remainder theorem.  Schoof's algorithm has subsequently been improved by 
Atkin~\citep{Atkin1988,Atkin1992} and Elkies~\citep{Elkies1998}, 
leading to a practical algorithm with runtime polynomial in $\log q$.  
It has also been generalised, for example to higher genus curves by 
Pila~\citep{Pila1990}, however, the runtime of these algorithms remains 
exponential in the genus.

Another group of algorithms applicable primarily to elliptic curves 
is based on the \emph{canonical lift} of a curve from~$\mathbf{F}_{q}$ 
to the ring~$\mathbf{Z}_{q}$, which is the ring of integers of the 
unique unramified extension of $\mathbf{Q}_{p}$ with residue field 
$\mathbf{F}_{q}$.  The first of these algorithms is Satoh's 
method~\citep{Satoh2000} presented in~2000, which has runtime 
polynomial in the extension degree~$\log_{p} q$ but linear in the prime~$p$.

Finally, there is a class of so-called $p$-adic algorithms based on 
Kedlaya's algorithm~\citep{Kedlaya2001}, which directly computes a 
$p$-adic approximation to the action of Frobenius on the rigid cohomology 
spaces associated to hyperelliptic curves in odd characteristic.  
For a fixed prime~$p$, the runtime of Kedlaya's original algorithm 
is polynomial in the extension degree~$\log_{p} q$ and the genus~$g$.  
However, there have been a number of improvements.  Harvey~\citep{Harvey2007} 
improved the roughly linear dependence on the prime from~$p$ to $\sqrt{p}$.  
Denef and Vercauteren~\citep{DenVer2006} included the case $p = 2$, and 
Castryck, Denef and Vercauteren~\citep{CasDenVer2006} extended the algorithm 
to so-called nondegenerate curves.

Lauder's \emph{deformation method} presented in~\citep{Lau04a} 
applies to smooth projective hypersurfaces, additionally making use 
of the theory of deformation due to Dwork~\citep{Dwork62b}.  The 
runtime of this algorithm is polynomial in the prime~$p$, the 
extension degree~$\log_{p} q$, and the genus~$g$.  Subsequently, 
Gerkmann~\citep{Gerkmann2007} improved a number of $p$-adic bounds 
and developed a better implementation.  Moreover, the practicality 
of the deformation method has also been demonstrated for various 
classes of curves.  Hubrechts~\citep{Hubrechts2007, Hubrechts2008} 
used this algorithm to compute the zeta function of hyperelliptic curves, 
Castryck, Hubrechts and Vercauteren~\citep{CasHubVer2008} considered 
a slightly more general class of $C_{a,b}$-curves, and 
Tuitman~\citep{Tuitman2011} extended this work to nondegenerate curves. 

Finally, we mention Lauder's \emph{fibration method} introduced 
in~\citep{Lauder2006}, which proceeds recursively on the dimension 
of a hypersurface.  This approach has been developed further 
by Walker~\citep{Walker2009}.

% This thesis %%%%%%%%%%%%%%%%%%%%%%%%%%%%%%%%%%%%%%%%%%%%%%%%%%%%%%%%%%%%%%%%%

\section{This thesis}

We present our results pertaining to the main goal of this thesis in 
Part~{I}, which is to demonstrate the practicality of Lauder's deformation 
method for a wide class of smooth projective hypersurfaces.  We build on 
the work of Lauder~\citep{Lau04a} and Gerkmann~\citep{Gerkmann2007}, 
resolving bottlenecks in previous computations by employing both theoretical 
advances as well as improved implementation details.  
In Chapter~\ref{ch:01-DMIntro}, we provide an overview of the deformation 
method and briefly revisit some of the underlying theoretical machinery 
and results.  As there are no original contributions in this chapter we 
omit many technical details and do not include any proofs.  
The deformation method relies on the computation of the action of 
Frobenius on a fibre in the family of hypersurfaces for which 
this computation is easy and we address this issue in 
Chapter~\ref{ch:01-DiagFrob}.  We present 
a novel algorithmic approach in the case of diagonal hypersurfaces, 
which is vastly superior to previous computations in practice, together 
with complementing theoretical results and a complexity analysis.
In Chapter~\ref{ch:GMConnection}, we consider another major step in the 
deformation method, namely the computation of the Gauss--Manin connection 
matrix.  Existing implementations~\citep{Lau04a,Gerkmann2007,Kedlaya2011} 
make use of Gr\"obner base computations for rational functions in order 
to carry out effective computations in de~Rham cohomology spaces, which 
leads to poor practical performance and limits the deformation method to 
hypersurfaces defined by sparse polynomials.  We reduce the impact 
of this step by treating it as an explicit finite-dimensional linear 
algebra problem.  We collect the remaining steps of the deformation 
method in Chapter~\ref{ch:01-Main}  as we do not present any original 
theoretical contributions.  Our practical improvements are due to 
theoretical advances by Kedlaya and Tuitman~\citep{KedlayaTuitman2012} 
and improved implementation details.  Finally, we present a number of 
examples in Chapter~\ref{ch:01-Exmp}, hoping to convince the reader of 
the practical relevance of this work.  In particular, in 
Section~\ref{sec:06-GenericQuarticSurface} we demonstrate that we can 
compute the zeta function of an arbitrary smooth projective quartic 
surface over a finite field of small characteristic in a matter of hours.

In Part~{II}, we present a constant factor improvement to the 
radix conversion of polynomials, which is a key step and the 
main bottleneck in current implementations~\citep{Lauder2006,Walker2009} 
of Lauder's fibration method.  Our brief review of the problem 
in Chapter~\ref{ch:02-Intro} is followed by a presentation of suitable 
algorithms in Chapter~\ref{ch:02-Algorithms}.  We observe a vast 
practical improvement when reconsidering an example due to Lauder 
in Chapter~\ref{ch:02-Exmp}.

The practical improvements that we have observed in the first part of 
this thesis rely on an efficient implementation of the underlying 
$p$-adic arithmetic, which we address in Part~{III}.  In fact, we 
go beyond treating merely the functions utilised in the deformation 
method and develop a complete module for unramified $p$-adic arithmetic, 
which we have implemented in the C library {\sc FLINT}~\citep{FLINT}.


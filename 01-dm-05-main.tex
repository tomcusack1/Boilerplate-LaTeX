% Chapter:  From the differential system to the zeta function %%%%%%%%%%%%%%%%%

In this chapter we describe the remaining steps of the deformation method:  
how to use the differential system governed by the Gauss--Manin connection 
matrix and the action of Frobenius on the initial fibre $\Hrig^{n}(U_0)$ 
to determine the action of Frobenius on the fibre~$\Hrig^{n}(U_{t_1})$ 
and hence the zeta function of the hypersurface~$X_{t_1}$.
We point out that there are essentially no original contributions from the 
author;  we present the work of Lauder~\citep{Lau04} and also follow the 
description by Gerkmann~\citep{Gerkmann2007}, focussing on the details 
relevant to explicit computations.  The improvements in practical runtimes 
that we observe in this part of the algorithm are due to improved estimates 
of certain $p$-adic valuations by Kedlaya~\citep{Kedlaya2010} and estimates 
of pole orders by Kedlaya and Tuitman~\citep{KedlayaTuitman2012} as well as 
implementation details.

%%%%%%%%%%%%%%%%%%%%%%%%%%%%%%%%%%%%%%%%%%%%%%%%%%%%%%%%%%%%%%%%%%%%%%%%%%%%%%%
\section{$p$-Adic differential system and local expansion of Frobenius}

Recall that we let $\Phi$ denote the matrix for the $\sigma$-semilinear action 
of~$p^{-1} \Frob_p$ on $\Hrig^{n}(U/S)$, where $\sigma$ is the standard lift of the 
$p$th-power Frobenius to~$\mathbf{Q}_q$ and~$\mathbf{Q}_q(t)$ sending 
$t \mapsto t^p$, and $M$ the matrix for the Leibniz-linear connection on 
$\HdR^n(\mathfrak{U}/\mathfrak{S})$.
As described for example by Gerkmann~\citep[\S 5]{Gerkmann2007}, these 
matrices satisfy the following differential system,
\begin{equation*}
\Bigl(\frac{d}{dt} + M\Bigr) \Phi = p t^{p-1} \Phi \sigma(M).
\end{equation*}
Our first goal, discussed in this section, is the computation of 
a power series expansion of~$\Phi$ around the origin, satisfying 
the above differential equation and the initial condition that~$\Phi_0$ 
be the matrix for the action of $p^{-1} \Frob_p$ on $\Hrig^n(U_0)$.
We begin by observing that, upon setting 
\begin{equation*}
\Phi = C \Phi_0 \sigma(C)^{-1},
\end{equation*}
the above differential system is equivalent to the homogeneous system 
\begin{equation} \label{eq:01-GMDE-Homogenous}
\Bigl(\frac{d}{dt} + M\Bigr) C = 0
\end{equation}
with the initial condition that $C(0)$ is the identity matrix.  Let us write 
\begin{equation*}
M(t) = \frac{B(t)}{r(t)}
\end{equation*}
where $B(t) = \sum_{i=0}^{\deg(B)} B_i t^i$ is a matrix over $\mathbf{Q}[t]$ 
with $\deg(B) = \max_{i,j} \deg(B_{ij})$ and $r(t) \in \mathbf{Z}[t]$ is the 
\emph{Macaulay resultant} for the polynomial~$P$.  We can now obtain a 
power series solution $C(t) = \sum_{i \geq 0} C_i t^i$ around $t=0$ for 
\begin{equation*}
r(t) \frac{dC}{dt} + B(t) C(t) = 0
\end{equation*}
using the recursion 
\begin{equation*}
C_{i+1} = \frac{-1}{r_0 (i+1)} \biggl(
    \sum_{j=\max{\{0,i-\deg(B)\}}}^i B_{i-j} C_j + 
    \sum_{j=\max{\{0,i-\deg(r)\}}+1}^i r_{i-j+1} j C_j \biggr).
\end{equation*}
where we recall from Notation~\ref{not:01-02-main} that by 
assumption $r_0 \neq 0 \pmod{p}$ and so, in particular, $r_0 \neq 0$.

From a computational perspective, we are only interested in a power 
series solution to some $p$-adic and $t$-adic precisions $N$ and $K_2$, 
respectively, as discussed in the next two sections.  As such we need 
to understand the loss of $p$-adic precision involved in the computation 
of $C_0, C_1, \dotsc, C_{K_2-1}$.  Specifically, let $D_0, D_1, \dotsc$ 
denote an approximation defined by $D_0 = I$ and 
\begin{equation*}
D_{i+1} = \frac{-1}{r_0 (i+1)} \biggl(
    \sum_{j=\max{\{0,i-\deg(B)\}}}^i B_{i-j} D_j + 
    \sum_{j=\max{\{0,i-\deg(r)\}}+1}^i r_{i-j+1} j D_j \biggr) + 
    p^{\tilde{N}} E_{i+1},
\end{equation*}
where $(E_i)_{i \geq 1}$ is a sequence of $p$-adically integral matrices. 

The relationship between the desired precision~$N$ and the working 
precision~$\tilde{N}$ is described by Lauder~\citep[Theorem~5.1]{Lauder2006}, 
but this can be improved significantly by including recent bounds on the 
valuation of $C_0, C_1, \dotsc$ by Kedlaya~\citep{Kedlaya2010}.

\begin{lem} \label{lem:01-FrobBound}
Recall that the matrix~$\Phi$ for the action of $p^{-1} \Frob_p$ 
on $\Hrig^{n}(U/S)$ has entries in 
$\mathbf{Q}_p\langle t, r(t)^{-1}\rangle^{\dagger}$, 
and let $\Phi_t$ denote the matrix for the action of $p^{-1} \Frob_p$ 
on $\Hrig^{n}(U_t)$ with entries in $\mathbf{Q}_p$.  Let us set 
\begin{align*}
r & = \ord_p((n-1)!), \\
s & = (n + 1) \floor{\log_p(n-1)}.
\end{align*}
and define  $\delta = r + s$.  The $p$-adic valuation of the 
matrices~$\Phi$ and $\Phi_t$ is then at least~$-\delta$.
\end{lem}

\begin{proof}
This follows from Gerkmann~\citep[Lemma~3.3]{Gerkmann2007}.
\end{proof}

\begin{thm} \label{thm:valC}
For all $i \geq 0$, the $p$-adic valuation of the matrix~$C_i$ 
can be bounded by 
\begin{equation*}
\ord_p(C_i) \geq - \bigl(2 \delta + (n - 1)\bigr) \ceil{\log_p i}.
\end{equation*}
\end{thm}

\begin{proof}
Let us recall from Lemma~\ref{lem:01-FrobBound} that 
$\ord_p(\Phi) \geq -\delta$, and hence $\ord_p(p^{n-1}\Phi^{-1}) \geq -\delta$ 
by Poincar\'e duality.  Thus, we see that 
\begin{equation*}
\ord_p(\Phi) + \ord_p(\Phi^{-1}) \geq - 2 \delta - (n-1)
\end{equation*}
and the bound follows now from Theorem~{18.3.3} by 
Kedlaya~\citep{Kedlaya2010}.
\end{proof}

\begin{rem}
Kedlaya~\citep[Remark~18.3.4]{Kedlaya2010} also includes another bound,
\begin{equation*}
\ord_p(C_i) \geq (b - 1) \ord_p(M) 
            - \bigl(2 \delta + (n - 1)\bigr) \floor{\log_p i},
\end{equation*}
which is better than the estimate used in the previous theorem whenever 
$\ord_p(M)$ is positive.
\end{rem}

\begin{thm}
Let the sequences of matrices $(C_i)$ and $(D_i)$ over $\mathbf{Q}$ and 
$\mathbf{Q}_p$, respectively, be defined as above.  Then, for all $i \geq 0$, 
\begin{equation*}
\ord_p(C_i - D_i) \geq 
    \tilde{N} - \Bigl(2 \bigl(2 \delta + (n-1)\bigr) + 1\Bigr) \ceil{\log_p i}.
\end{equation*}
\end{thm}

\begin{proof}
This follows analogously to the result of 
Lauder~\citep[Theorem~5.1]{Lauder2006}.  
Indeed, its proof shows more specifically that 
\begin{equation*}
\ord_p(C_i - D_i) \geq 
    \tilde{N} + \min_{k+\ell=i} \Bigl(\ord_p(C_k) + 
                                      \ord_p(\ell^{-1} C_{\ell-1}^{-1})\Bigr).
\end{equation*}
We use the bound from Theorem~\ref{thm:valC} and observe that they 
also apply to the inverse matrix~$C^{-1}$, as this matrix satisfies 
the related differential equation 
\begin{equation} \label{eq:01-GMDE-Dual}
\Bigl(\frac{d}{dt} - M^t\Bigr) \bigl(C^{-1}\bigr)^t = 0.
\end{equation}
The result now follows.
\end{proof}

\begin{rem}
In order to determine the power series expansion of the matrix~$\Phi$, 
we also need to compute the matrix $\sigma(C)^{-1}$.  As we assume that 
the family~$\mathcal{X}$ of hypersurfaces is defined over $\mathbf{Z}_p$, 
the connection matrix~$M$ is a matrix over $\mathbf{Q}_p[t,r(t)^{-1}]$ 
and the local solution~$C$ has entries in~$\mathbf{Q}_p[[t]]$.  
As $\sigma$ acts trivially on~$\mathbf{Q}_p$, we have that 
$\sigma(C)^{-1} = C(t^p)^{-1} = C^{-1}(t^p)$. 

The matrix~$C^{-1}$ could be computed using matrix inversion 
over the ring $\mathbf{Q}_p[[t]]$.  An alternative approach 
follows from observing that whenever $C$ satisfies 
equation~\eqref{eq:01-GMDE-Homogenous}, its inverse~$C^{-1}$ 
satisfies the related differential equation~\eqref{eq:01-GMDE-Dual}. 
In practice, solving this equation for $C^{-1}$ is typically favourable 
compared to inverting the matrix~$C$.
\end{rem}

With $p$-adic approximations to the power series modulo $t^{K_2}$ available 
for $C$, $\Phi_0$, and $\sigma(C)^{-1}$, we are in a position to determine 
the power series for $\Phi = C \Phi_0 \sigma(C)^{-1}$.  The final issue 
to address in this section is the $p$-adic precision that we require for 
the three matrices $C$, $\Phi_0$, and $C^{-1}$.
We begin by developing generic bounds on the products of $p$-adic numbers 
and matrices.

\begin{prop} \label{prop:productval}
Let $x_1, \dotsc, x_{\ell} \in \mathbf{Q}_p$, where $\ell \geq 2$, 
be given as $x_i = p^{v_i} u_i$ with $v_i = \ord_p(x_i) \in \mathbf{Z}$ 
and $u_i \in \mathbf{Z}_p^{\times}$ whenever $x_i \neq 0$ and 
$u_i = v_i = 0$ otherwise.  Suppose that $N \in \mathbf{Z}$ is given 
such that $N > \sum_{j=1}^{\ell} v_j$ and, for all $i$, $N > \sum_{j \neq i} v_j$.

Let $\tilde{x}_1, \dotsc, \tilde{x}_{\ell}$ be $p$-adic approximations 
satisfying $\ord_p(x_i - \tilde{x}_i) \geq N - \sum_{j \neq i} v_j$ 
for all $i$.  Then 
\begin{equation*}
\ord_p(x_1 \dotsm x_{\ell} - \tilde{x}_1 \dotsm \tilde{x}_{\ell}) \geq N.
\end{equation*}
\end{prop}

\begin{proof}
First, note that $x_i \neq 0$ implies that $\tilde{x}_i \neq 0$. 
Otherwise, if $\tilde{x}_i = 0$ then
\begin{equation*}
N - \sum_{j \neq i} v_j \leq \ord_p(x_i - \tilde{x}_i) = \ord_p(x_i) = v_i
\end{equation*}
which implies that $N \leq \sum v_j$, a contradiction.

Represent $\tilde{x}_i$ as $p^{\tilde{v}_i} \tilde{u}_i$ with the 
same convention that $\tilde{u}_i \in \mathbf{Z}_p^{\times}$ unless 
$\tilde{x}_i = 0$, in which case $\tilde{u}_i = \tilde{v}_i = 0$.  
We first show that $v_i = \tilde{v}_i$ whenever $x_i \neq 0$.  Indeed, 
if $v_i \neq \tilde{v}_i$ then 
\begin{equation*}
\ord_p(x_i - \tilde{x}_i) = \min\{\ord_p(x_i), \ord_p(\tilde{x}_i)\} \leq \ord_p(x_i) = v_i
\end{equation*}
whereas from the assumptions we derive 
\begin{equation*}
\ord_p(x_i - \tilde{x}_i) \geq N - \sum_{j \neq i} v_j > v_i,
\end{equation*}
a contradiction.

Note that, for all $i$ such that $x_i \neq 0$, the assumption 
$\ord_p(x_i - \tilde{x}_i) \geq N - \sum_{j \neq i} v_j$ implies that 
$\ord_p(u_i - \tilde{u}_i) \geq N - \sum_j v_j$.  Specifically, 
\begin{equation*}
\ord_p(u_i - \tilde{u}_i) = \ord_p(x_i - \tilde{x}_i) - v_i \geq N - \sum_{j=1}^{\ell} v_j.
\end{equation*}

Considering the product $x_1 \dotsm x_{\ell}$, we distinguish two 
cases.  First, assume that $x_i \neq 0$ for all $i$.  We find that 
\begin{equation*}
\ord_p(x_1 \dotsm x_{\ell} - \tilde{x}_1 \dotsm \tilde{x}_{\ell}) = 
    \sum_{j} v_j + \ord_p(u_1 \dotsm u_{\ell} - \tilde{u}_1 \dotsm \tilde{u}_{\ell}) \geq N.
\end{equation*}
In the other case, suppose that for some $k < \ell$, $x_1, \dotsc, x_k$ 
are non-zero and that $x_{k+1}, \dotsc, x_{\ell}$ are all zero.  In 
particular, $v_{k+1} = \dotsb = v_{\ell} = 0$.  Then 
\begin{equation*}
\begin{split}
\ord_p(x_1 \dotsm x_{\ell} - \tilde{x}_1 \dotsm \tilde{x}_{\ell}) 
  & = \ord_p(\tilde{x}_1 \dotsm \tilde{x}_{\ell}) \\
  & \geq v_1 + \dotsb + v_k + N - \sum_{j \neq k+1} v_j + \dotsb + N - \sum_{j \neq \ell} v_j \\
  & = (\ell - k) N - (\ell - k - 1) (v_1 + \dotsb + v_k) \\
  & > N. \qedhere
\end{split}
\end{equation*}
\end{proof}

\begin{prop} \label{prop:matrixproductval}
Let $A_1, \dotsc, A_{\ell}$ be $b \times b$ matrices over $\mathbf{Q}_p$, 
where $\ell \geq 2$, given as $A_i = p^{v_i} U_i$ with 
$v_i = \ord_p(A_i) \in \mathbf{Z}$ and at least one entry of $U_i$ a 
$p$-adic unit $A_i \neq 0$ and $v_i = 0$ and $U_i$ the zero matrix 
otherwise.  Suppose that $N \in \mathbf{Z}$ is given such that 
$N > \sum_{j=1}^{\ell} v_j$ and, for all $i$, $N > \sum_{j \neq i} v_j$.

Let $\tilde{A}_1, \dotsc, \tilde{A}_{\ell}$ be $p$-adic approximations 
satisfying $\ord_p(A_i - \tilde{A}_i) \geq N - \sum_{j \neq i} v_j$ 
for all $i$.  Then 
\begin{equation*}
\ord_p(A_1 \dotsm A_{\ell} - \tilde{A}_1 \dotsm \tilde{A}_{\ell}) \geq N.
\end{equation*}
\end{prop}

\begin{proof}
We can follow the proof of Proposition~\ref{prop:productval}, 
observing that, for matrices $A$, $B$ over $\mathbf{Q}_p$, 
we have $\ord_p(A + B) \geq \min \{\ord_p(A), \ord_p(B)\}$.
\end{proof}

\begin{cor}
Suppose that $K_2, N_2$ are positive integers such that $N_2$ is greater than 
the sum of two or three of $\ord_p(\Phi_0)$, $\ord_p\bigl(C \bmod t^{K_2}\bigr)$, 
and $\ord_p\bigl(C^{-1} \bmod{t^{\ceil{K_2/p}}}\bigr)$.
In order to compute the power series expansion around the origin 
of the matrix~$\Phi$ modulo~$t^{K_2}$ to $p$-adic precision~$N_2$, it 
suffices to compute the matrices $C$, $C^{-1}$ and $\Phi_0$ to 
$p$-adic precision $N_3$, $N_3'$ and $N_4$, respectively, where 
\begin{align*}
N_3  & \geq N_2 + \bigl(2\delta+(n-1)\bigr) \ceil{\log_p \ceil{K_2/p}} + \delta, \\
N_3' & \geq N_2 + \bigl(2\delta+(n-1)\bigr) \ceil{\log_p K_2} + \delta, \\
N_4  & \geq N_2 + \bigl(2\delta+(n-1)\bigr) \bigl(\ceil{\log_p K_2} + \ceil{\log_p \ceil{K_2/p}}\bigr).
\end{align*}
\end{cor}

\begin{proof}
This follows from Proposition~\ref{prop:matrixproductval}, the bounds 
on the valuations of $C$ and $C^{-1}$ from Theorem~\ref{thm:valC} and 
the bound $\ord_p(\Phi_0) \geq - \delta$ from Lemma~\ref{lem:01-FrobBound}.
\end{proof}

%%%%%%%%%%%%%%%%%%%%%%%%%%%%%%%%%%%%%%%%%%%%%%%%%%%%%%%%%%%%%%%%%%%%%%%%%%%%%%%
\section{Analytic continuation and evaluation}
\label{sec:01-dm-06-continuation}

In the previous section we described how to compute the matrix~$\Phi$ 
for the action of $p^{-1} \Frob_p$ on $\Hrig^{n}(U/S)$ 
as a local power series expansion around $t = 0$.  In order to evaluate it 
at a second point $t = t_1$, we wish to $p$-adically approximate it 
using rational functions with denominators a power of $r(t)$, which 
is possible as the entries of the matrix~$\Phi$ are \emph{overconvergent 
functions} as defined in Definition~\ref{defn:Overconvergence} and 
described e.g.\ by Lauder~\citep[\S 3.5]{Lauder2006}.

This step, the analytic continuation of the local expansion, is described 
by Lauder in~\citep[\S 5.2]{Lauder2006} and the computational details are 
explained in~\citep[\S 8.1]{Lau04a}.  Combining this step with the subsequent 
evaluation, our specific problem is the following.  Given a desired $p$-adic 
precision~$N_2$, we would like to determine integers~$K_1$ and~$K_2$ such 
that, modulo~$p^{N_2}$, the entries of the matrix $\Psi(t) = r(t)^m \Phi(t)$ 
truncated modulo~$t^{K_2}$ allow us to compute the matrix $\Phi_{t_1}$ for 
the action of $p^{-1} F_p$ on $\HdR^{n}(\mathfrak{U}_{t_1})$ as 
$r(\hat{t_1})^{-1} \Psi(\hat{t_1})$ modulo~$p^{N_2}$.

Besides the work of Lauder~\citep[\S 8.1]{Lau04a}, Gerkmann also described 
suitable results in~\citep[\S 6]{Gerkmann2007}, however the estimates are 
not sharp in practice.  There has been recent progress by Kedlaya and 
Tuitman~\citep[Theorem~2.1]{KedlayaTuitman2012}, which we present in a 
slightly simplified and weakened form:

\begin{thm} \label{thm:KedlayaTuitman}
Let $\mathfrak{Z}$ denote the complement of the open dense 
subscheme $\mathfrak{S}$ of $\mathbf{P}^{1}(\mathbf{Q}_q)$ 
and let $z$ be an unramified geometric point of $\mathfrak{Z}$ 
such that $\mathfrak{Z}$ does not contain any other points 
with the same reduction modulo~$p$.  For a fixed a basis for 
$\Hrig^n(U/S)$, suppose that the matrix 
for the connection~$\nabla$ has at most a simple pole at~$z$ 
and that the exponents $\lambda_1, \dotsc, \lambda_{b}$, 
which are defined as the eigenvalues of $(t - z) M \vert_{t=z}$ 
and known to be rational numbers, have non-negative $p$-adic 
valuation.  Then the matrix~$\Phi$ for the action of $p^{-1} \Frob_p$ 
has a pole at $z$ of order at most 
\begin{equation} \label{eq:KedlayaTuitman}
\max_{i} \lambda_i - p \min_{i} \lambda_i + p g(N)
\end{equation}
where $g(N)$ is defined by 
\begin{equation}
g(N) = \max\Bigl\{ i \in \mathbf{N} : i - \delta - \bigl(2\delta + (n-1)\bigr) \ceil{\log_p i} < N \Bigr\}.
\end{equation}
\end{thm}

\begin{proof}
See \citep[Theorem~2.1]{KedlayaTuitman2012}.
\end{proof}

\begin{rem} \label{rem:KedlayaTuitman}
In practice, it might be convenient to avoid computing the exponents and 
verifying the hypotheses of the previous theorem.  This is particularly 
relevant since even when the hypotheses are not satisfied, the required 
bounds are often closer to the result of Kedlaya and Tuitman than to the 
estimates provided by Gerkmann.

The contribution of the terms $\max_i \lambda_i - p \min_i \lambda_i$ 
in equation~\eqref{eq:KedlayaTuitman} typically is small compared to the 
term~$p g(N_2)$. Therefore, in a heuristic implementation, one 
could choose e.g.\ $K_1 = 1.1 \times p N_2$ and $K_2 = (\deg(r) + 1) K_1$.
\end{rem}

Having obtained an expression for the action of~$p^{-1} \Frob_p$ 
on~$\Hrig^{n}(U_{t_1})$ in terms of rational functions, 
we conclude this section by describing the evaluation at the fibre 
$t = t_1$.  Modulo~$p^{N_2}$, the matrix $\Phi_{t_1}$ representing 
$p^{-1} \Frob_p$ on $\Hrig^{n}(U_{t_1})$ is equal to 
\begin{equation}
\Phi_{t_1} = 
    r(\hat{t}_1)^{-K_1} 
    \Bigl( r(t)^{K_1} \Phi(t) \bmod{t^{K_2}} \Bigr) \Big\vert_{t=\hat{t}_1} \pmod{p^{N_2}}.
\end{equation}
Since, by assumption, $r(t_1)$ is a $p$-adic unit and hence so is 
$r(\hat{t}_1)^{-K_1}$ we observe that it suffices to compute the 
Teichm\"uller lift~$\hat{t}_1$ and the matrix~$r(t)^{K_1} \Phi(t)$ 
over~$\mathbf{Q}_p[t]$ to $p$-adic precision~$N_2$.

%%%%%%%%%%%%%%%%%%%%%%%%%%%%%%%%%%%%%%%%%%%%%%%%%%%%%%%%%%%%%%%%%%%%%%%%%%%%%%%
\section{Recovering zeta functions}

Recall that the zeta function of the hypersurface~$X_{t_1}$ is of the form,
\begin{equation*}
Z(X_{t_1},T) = \frac{p(T)^{(-1)^n}}{(1 - T) (1 - qT) \dotsm (1 - q^{n-1}T)}
\end{equation*}
where $p(T) = \det \bigl( 1 - T q^{-1} \Frob_q | \HdR^n(\mathfrak{U}_{t_1}) \bigr)$ 
is a polynomial defined over the integers.

Thus, the remaining two steps in the deformation method are as follows.  
Firstly, we obtain the matrix of the $q$th-power Frobenius from the matrix of 
the $p$th-power Frobenius.  Secondly, we compute its reverse characteristic 
polynomial to suitable \mbox{$p$-adic} precision in order to recover the zeta 
function.

\subsection{Computing the action of $q^{-1} \Frob_q$ on $\Hrig^{n}(U_{t_1})$}

Let $\Phi_{t_1}$ and $\Phi_{t_1}^{(q)}$ denote the matrices representing the 
actions of $p^{-1} \Frob_p$ and $q^{-1} \Frob_q$ on~$\Hrig^{n}(U_{t_1})$, 
respectively.  As the action of $p^{-1} \Frob_p$ is $\sigma$-semilinear, we have 
that 
\begin{equation*}
\Phi_{t_1}^{(q)} = 
    \Phi_{t_1} \sigma(\Phi_{t_1}) \dotsm \sigma^{a-1}(\Phi_{t_1}),
\end{equation*}
where $a = \log_p q$.  Note that the lift of Frobenius~$\sigma$ 
is valuation preserving and hence the valuations of the matrices 
$\Phi_{t_1}, \Phi_{t_1}^{\sigma}, \dotsc, \Phi_{t_1}^{\sigma^{a-1}}$ 
are at least $-\delta$ by Lemma~\ref{lem:01-FrobBound}.

It now follows from Proposition~\ref{prop:matrixproductval} that 
it suffices to compute the matrix~$\Phi_{t_1}$ to \mbox{$p$-adic} 
precision~$N_2$ at least \mbox{$N_1 + (a-1) \delta$} in order to 
determine $\Phi_{t_1}^{(q)}$ to \mbox{$p$-adic} precision~$N_1$.

\subsection{Computing characteristic polynomials}

In this section we address the precision loss when computing the reverse 
characteristic polynomial $\det(1 - t A)$ of a $b \times b$ matrix~$A$ 
given to finite precision over $\mathbf{Q}_p$.

In general, from the definition of the determinant function and 
Proposition~\ref{prop:productval}, it appears that the precision 
loss could be as great as $(b-1) \ord_p(A)$.  However, in the case 
of matrices representing the action of Frobenius 
on $\Hrig^n(U_{t_1})$, much better bounds are available.

\begin{lem} \label{lem:charpoly}
Define $\delta = \delta(n,p)$ as in Lemma~\ref{lem:01-FrobBound} and 
suppose that $\tilde{\Phi}_{t_1}^{(q)}$ is an approximation to 
$\Phi_{t_1}^{(q)}$ with 
$\ord_p\bigl(\Phi_{t_1}^{(q)}-\tilde{\Phi}_{t_1}^{(q)}\bigr) \geq N + \delta$.
Then 
\begin{equation*}
\ord_p \Bigl( \det\bigl(1 - t \Phi_{t_1}^{(q)}\bigr) 
            - \det\bigl(1 - t \tilde{\Phi}_{t_1}^{(q)}\bigr) \Bigr) \geq N.
\end{equation*}
\end{lem}

\begin{proof} 
See Gerkmann~\citep[Lemma~3.3, Lemma~3.4]{Gerkmann2007}.
\end{proof}

The previous lemma allows us to take $N_1 = N_0 + \delta$ in our 
description of the deformation algorithm.

\subsection{Computing Weil polynomials}

In the final step of the deformation method, we compute the 
reverse characteristic polynomial $p(T)$ of the matrix $\Phi_{t_1}^{(q)}$, 
which represents $q^{-1} \Frob_q$ on $\Hrig^n(U_{t_1})$, to some finite 
$p$-adic precision.  Which precision~$N_0$ is necessary in order to 
correctly recover the polynomial~$p(T)$ over the integers?

\begin{thm} \label{thm:N0}
In order to recover $p(T)$ over $\mathbf{Z}$ it suffices to compute 
an approximation modulo~$p^N$ where 
\begin{equation*}
p^N > 2 \max_{0 \leq i \leq b} \binom{b}{i} q^{i (n-1) / 2}.
\end{equation*}
Moreover, this can be improved to 
\begin{equation*}
p^N > 2 \binom{b}{\floor{b/2}} q^{\floor{b/2} (n-1) / 2}
\end{equation*}
provided that the sign $\epsilon = \sgn(\det(F_q))$ of the 
functional equation for the zeta function is known.
\end{thm}

\begin{proof}
This is a slight reformulation of {Theorem~3.2} in~\citep{Gerkmann2007}, 
which follows readily from its proof.
\end{proof}

\begin{rem}
We observe that the sign $\epsilon = \sgn(\det(q^{-1} \Frob_q)) = 1$ 
whenever $n$ is even.  This can be seen from the statement of the 
Weil conjectures in Theorem~\ref{thm:01-Zetafunctions}.
\end{rem}

\begin{rem}
In the case that $n$ is odd, the sign~$\epsilon$ of the functional 
equation is unknown a priori.  However, in practice it is often still 
possible to avoid having to use the greater of the two precisions in 
Theorem~\ref{thm:N0} by just computing one additional coefficient 
exactly.  Writing $p(T) = \sum_{i=0}^{b} a_i T^i$, we can recover 
$a_0, \dotsc, a_{\floor{b/2}+1}$ exactly provided 
\begin{equation*}
p^N > 2 \max\biggl\{\binom{b}{\floor{b/2}} q^{\floor{b/2} (n-1) / 2}, 
                   \binom{b}{\floor{b/2}+1} q^{(\floor{b/2}+1) (n-1)/2} \biggr\}.
\end{equation*}
This allows us to determine the sign~$\epsilon$ from the two 
coefficients $a_{\ceil{b/2}-1}$ and $a_{\floor{b/2}+1}$, provided 
that they are non-zero, and we can then recover the remaining 
coefficients using the functional equation.
\end{rem}

\begin{rem} \label{rem:N0Surfaces}
In the case of smooth projective surfaces, when $p > 2$ and subject to 
certain technical conditions, we can often exploit the growing divisibility 
of the coefficients of $p(T)$ ensured by the Hodge polygon.  For such a surface 
of degree~$d$, we know that the Hodge numbers $h_{0,2}$, $h_{1,1}$ and $h_{2,0}$ 
satisfy $h_{0,2} = h_{2,0} = \binom{d-1}{3}$ and $2 h_{0,2} + h_{1,1} = b$. 
It is now easier to determine the integer coefficients of the 
polynomial~$q^{h_{0,2}} p(T/q)$ as the roots of the polynomial $p(T/q)$ lie 
on the unit circle.  This allows us to take 
\begin{align*}
N_0 & = a h_{0,2} + \floor{\log_p \biggl( 2 \binom{b}{\floor{b/2}}\biggr)} + 1,\\
N_1 & = N_0 + a,
\end{align*}
where $N_0$ here refers to the required precision for $q^{h_{0,2}} p(T/q)$ 
and $N_1$ as before refers to the precision required for the matrix 
representing $q^{-1} \Frob_q$ on $\Hrig^{n}(U_{t_1})$.
For further details, we refer the reader to 
Lauder~\citep[\S 9.3.2, Proposition~9.6]{Lauder2006}.
\end{rem}

